\documentclass{beamer}
\mode<presentation> {
\usetheme{Antibes}
\definecolor{zelena}{rgb}{.9,.74,.00}
\usecolortheme[named=zelena]{structure}
}

\usepackage[T1]{fontenc}

\usepackage{hyperref} 
\usepackage{graphicx}
\usepackage{color}
\usepackage[english,serbian]{babel}
\usepackage[utf8]{inputenc}
\usepackage{listings}


\def\d{{\fontencoding{T1}\selectfont\dj}}
\def\D{{\fontencoding{T1}\selectfont\DJ}}

\title[Prepoznavanje saobraćajnih znakova pomoću CNN]{Prepoznavanje saobraćajnih znakova pomoću CNN}

\author{Jana Jovičić, Jovana Pejkić}
\institute[Matematički fakultet]
{
\small{Prezentacija seminarskog rada \\u okviru kursa\\ Računarska inteligencija \\ Matematički fakultet\\}
\medskip
\textit{jana.jovicic755@gmail.com, jov4ana@gmail.com}
}
\date{}

\begin{document}


\begin{frame}
\titlepage
\end{frame}

%------------------------------------------------


\begin{frame}
\frametitle{Sadržaj}
\tableofcontents
\end{frame}

%------------------------------------------------



\section{Cilj rada}
\begin{frame} 
\frametitle{Cilj rada}

\begin{itemize}
\item Za bazu podataka kineskih saobracajnih znakova izvršiti što precizniju klasifikaciju
\item Implementirati CNN u programskom jeziku Python uz korišćenje Keras biblioteke
\item Isprobati nekoliko različitih arhitektura mreže
\item Uporediti dobijene rezultate i izvesti zakljlučke
\end{itemize}


%\begin{figure}
%\includegraphics[width=270pt, height=126pt]{ime_slike.jpg}
%\caption{Naslov slike}
%\end{figure}

\end{frame}

%------------------------------------------------

\section{Informacije o korišćenom skupu podataka}
\begin{frame}
\frametitle{Informacije o korišćenom skupu podataka}

\begin{itemize}
\item Baza sadrži \textbf{6164 slika} saobraćajnih znakova
\begin{itemize}
\item podeljenih u \textbf{58 klasa}
\item pri čemu \textbf{trening skup} sadrži \textbf{4170 slika}
\item a \textbf{test skup 1994 slika}
\end{itemize}
\item Zbog nejednakog broja slika (negde 5, negde 400) po klasama, korišćen je deo baze
\item Izdvojeno je \textbf{10 klasa} koje su imale priblizno jednak broj slika
\item Dobijen je \textbf{trening skup} od \textbf{1693 slika} i \textbf{test skup} od \textbf{764 slika}


% TODO TODO TODO TODO TODO
% Na slici 10 je prikazan po jedan znak iz svake klase trening skupa, zajedno sa brojem elemenata te klase.
% Podaci o test skupu, mogu se videti na slici.
% Ubaciti slike pa srediti ovo zakomentarisano!

\end{itemize}

\end{frame}


%------------------------------------------------

%------------------------------------------------


\subsection{Unutrašnja struktura CNN}
\begin{frame}
\frametitle{Unutrašnja struktura CNN}

\begin{itemize}
\item \textbf{Konvolutivni sloj}
\begin{itemize}
\item Operacija konvolucije: $$(f * g)_{ij} = \sum_{k=0}^{p-1} \sum_{l=0}^{q-1} f_{i-k, i-l}*g_{k, l}$$
\end{itemize}
\item \textbf{Sloj agregacije}
\begin{itemize}
\item Maxpooling / Averagepooling
\end{itemize}
\item \textbf{Potpuno povezani sloj}
\begin{itemize}
\item Softmax funkcija
\item Categorical crossentropy
\end{itemize}
\end{itemize}

\begin{figure}
\includegraphics[scale=0.5]{cnn_layers.jpg}
\caption{Struktura konvolutivne neuronske mreže}
\end{figure}

\end{frame}



\subsection{Filtriranje i proširivanje}
\begin{frame}
\frametitle{Filtriranje, proširivanje i korak}

\begin{itemize}
\item Vrše se na konvolutivnom sloju
\item \textbf{Filtriranje} je ,,ekstraktovanje" karakteristika ulaza (slike)
\begin{itemize}
\item tako što se izvršava operacija konvolucije
\end{itemize}
\item Dimenzija mape nakon filtriranja je manja od dimenzije ulaza
\item Vrši se \textbf{proširivanje} ulazne matrice
\begin{itemize}
\item nulama / vrednostima koje su već na obodu
\end{itemize}
\item \textbf{Korak} - za koliko piksela se filter pomera duž slike
\end{itemize}

\begin{figure}
\includegraphics[scale=0.5]{filtriranje_mape.png}
\caption{Filtriranje}
\end{figure}


%\begin{Primer proširivanja ulazne matrice nulama}
%\includegraphics[scale=0.3]{padding.png}
%\caption{Primer proširivanja ulazne matrice nulama}
%\end{figure}

\end{frame}



\subsection{Aktivaciona funkcija}
\begin{frame}
\frametitle{Aktivaciona funkcija}

\begin{itemize}
\item Funkcija kojom se ograničavaju vrednosti izlaza neurona
\begin{itemize}
\item opseg izlaza neurona obično je u intervalu [0,1] ili [-1,1]
\end{itemize}
\item Više vrsta aktivacionih funkcija:
\begin{itemize}
\item \textbf{ReLU} (Rectified Linear Unit): $f(z)=max(0, z)$
\item \textbf{tanh}: $\tanh(x) = \dfrac{e^{2x}-1}{e^{2x}+1}$
\item \textbf{sigmoid}: $\sigma(x) = \dfrac{1}{1 + e^{-x}}$
\end{itemize}
\end{itemize}

%\begin{figure}
%\includegraphics[scale=0]{relu_graph.png}
%\caption{Izgled ReLU funkcije}
%\end{figure}

\begin{figure}
\includegraphics[scale=0.38]{graphs_prez.png}
\caption{Grafici funkcija: ReLU, tanh i sigmoid}
\end{figure}

\end{frame}


\subsection{Agregacija metodom Maxpool}
\begin{frame}
\frametitle{Agregacija metodom Maxpool}

\begin{itemize}
\item Jedan od metoda koji se koristi na sloju za agregaciju, najzastupljeniji
\item Vraća \textbf{maksimum} dela slike prekrivene filterom 
\item Ideja je da se informacije o slici što više ,,ukrupne"
\end{itemize}

\begin{figure}
\includegraphics[scale=0.5]{maxpool.jpeg}
\caption{Metod Maxpool sa filterom [2x2] i korakom veličine 2}
\end{figure}

\end{frame}


% ------------------- Ovde treba potpuno povezani sloj ------------------
% ------------------- Ovde treba potpuno povezani sloj ------------------
% ------------------- Ovde treba potpuno povezani sloj ------------------
% ------------------- Ovde treba potpuno povezani sloj ------------------
% ------------------- Ovde treba potpuno povezani sloj ------------------
% ------------------- Ovde treba potpuno povezani sloj ------------------
% ------------------- Ovde treba potpuno povezani sloj ------------------
% ------------------- Ovde treba potpuno povezani sloj ------------------
% ------------------- Ovde treba potpuno povezani sloj ------------------
% ------------------- Ovde treba potpuno povezani sloj ------------------





%------------------------------------------------
%------------------------------------------------
%------------------------------------------------
%------------------------------------------------
%------------------------------------------------

\section{Modeli}

\subsection{Model 1}
\begin{frame}
\frametitle{Model 1}

\begin{itemize}

\item Jedan od prvih modela koji je imao uspeha nad test podacima
\item Sastoji se iz:
\begin{itemize}
\item 4 konvolutivna sloja
\item 2 agregirajuca sloja
\item 2 potpuno povezana sloja
\end{itemize}

\item U svim konvolutivnim slojevima:
\begin{itemize}
\item velicina jezgra je 3x3
\item broj ltera na izlazu iz konvolucije je 32
\end{itemize}

% TODO TODO TODO TODO TODO Proveriti da li u svakom sloju?
% TODO TODO TODO TODO TODO Sta u svakom?

\item U svakom sloju se koristi \textbf{ReLU} aktivaciona funkcija
\item Agregacija se vrsi biranjem \textbf{maksimalne vrednosti} dela mape karakteristika koji je prekriven lterom

\end{itemize}


\end{frame}


\subsection{Model 1 nastavak}
\begin{frame}
\frametitle{Model 1 nastavak}

\begin{itemize}

\item Funkcijom \textbf{Dropoup()} je iskljucivan odreden broj nasumicno odabranih neurona (da bi se sprecilo preprilagodjavanje)
\item Nakon agregacija je iskljuceno 20\% neurona, pre FC sloja 50\%

\item Poslednji potpuno povezani (FC) sloj
\begin{itemize}
\item ima onoliko neurona koliko ima klasa
\item koristi softmax aktivacionu funkciju
\end{itemize}

\item Ucenje modela je sprovedeno u 30 epoha
\item Batch size je postavljen na 32
\begin{itemize}
\item sto znaci da u svakoj iteraciji uzima 32 primerka iz trening skupa koja ce biti propagirana kroz mrezu
\end{itemize}

\item Optimizacija modela je izvrsena pomocu \textbf{gradijentnog spusta}

\end{itemize}

\end{frame}

%------------------------------------------------

\subsection{Model 2}
\begin{frame}
\frametitle{Model 2}


\end{frame}

%------------------------------------------------


\subsection{Model 2 nastavak}
\begin{frame}
\frametitle{Model 2 nastavak}

\begin{itemize}
\item
\item
\item
\end{itemize}

\end{frame}

%------------------------------------------------
%------------------------------------------------
%------------------------------------------------
%------------------------------------------------
%------------------------------------------------


\subsection{Model 3: AlexNet}
\begin{frame}
\frametitle{Model 3: AlexNet}

\begin{itemize}
\item AlexNet arhitektura je jedna od prvih dubokih mreža
\item Sastoji se iz:
\begin{itemize}
\item 5 konvolutivna sloja
\item 3 potpuno povezana sloja
\end{itemize}
\item Kao aktivaciona funkcija koristi se ReLu
\end{itemize}


\begin{figure}
\includegraphics[scale=0.55]{alexnet_arh.png}
\caption{AlexNet arhitektura}
\end{figure}


\end{frame}



\subsection{Model 3: AlexNet - nastavak}
\begin{frame}
\frametitle{Model 3: AlexNet - nastavak}

\begin{itemize}
\item Mreža koju smo implementirale svaki izlaz iz konvolutivnog sloja normalizuje pre nego što ga prosledi narednom sloju
\item Pre normalizacije (nakon 1., 2. i 5. konv. sloja) - agregacija
\begin{itemize}
\item sa parametrom padding = 'valid' (nema proširenja)
\end{itemize}
\item Poravnavajući sloj ("ispravlja" mapu karakteristika u vektor)
\item 3 Dense sloja + Dropout()
\begin{itemize}
\item Dropout() sprečava preprilagodavanje modela
\end{itemize}
\item Na kraju je izlazni sloj koji preslikava ulaz u zadati broj klasa
\begin{itemize}
\item kao funkciju aktivacije koristi Softmax.
\end{itemize}
\end{itemize}

%\begin{figure}
%\includegraphics[scale=0.5]{confussion_matrix_alexNet_1000epoh.png}
%\caption{Matrica konfuzije i ostale statistike AlexNet modela za 1000 epoha}
%\end{figure}

\end{frame}


\subsection{Model 3: AlexNet - statistike}
\begin{frame}
\frametitle{Model 3: AlexNet - statistike}

\begin{itemize}
\item Kod AlexNet modela, vreme izvršavanja je oko 10 puta veće za 100 nego za 10 epoha

\item Najmanja preciznost za AlexNet model se postiže za funkciju sigmoid, 10 epoha i Batch size 256, a iznosi 0.696

\begin{itemize}
\item to je jedini slučaj da je preciznost ispod 0.71
\item uglavnom su preciznosti u intervalu [0.801, 0.934]
\item ne osciluje mnogo izvan pomenutog intervala
\end{itemize}

\item Prosečna preciznost za 10 epoha je 82\%, a za 100 epoha 90\%

\item Iz ovoga se može zaključiti da je za AlexNet model i dati skup podataka
bolje trenirati model u što više epoha
\item Ovo potvrduje i činjenica da je u 1000 epoha ovaj model dostigao najveću preciznost od 95\% (veću od sva tri modela)

\end{itemize}

\end{frame}


\subsection{Model 3: AlexNet - statistike - nastavak}
\begin{frame}
\frametitle{Model 3: AlexNet - statistike - nastavak}

\begin{figure}
\includegraphics[scale=0.42]{alexNet_tabela.png}
\caption{Rezultati za AlexNet model za različite vrednosti broja epoha (10 i 100), funkcije aktivacije (relu, tanh i sigmoid) i Batch size (32, 64 i 256)}
\end{figure}

\end{frame}



\subsection{AlexNet i LeNet-5}
\begin{frame}
\frametitle{AlexNet i LeNet-5}

\begin{itemize}
\item Primetiti da se vrednosti na y-osi ova dva grafika razlikuju
\item To je zato što LeNet-5 daje malu preciznost za f-ju sigmoid
\item Maksimalna preciznost AlexNet mreže je 0.94, a LeNet-5 0.9

\end{itemize}

\begin{figure}
\includegraphics[scale=0.45]{graphs_alexnet_lenet.png}
\caption{Grački prikaz preciznosti za mreže AlexNet i LeNet-5 za 100 epoha, za funkcije sigmoid, tanh i ReLu}
\end{figure}

\end{frame}


%------------------------------------------------

%------------------------------------------------





\section{Zaključak}

\begin{frame}
\frametitle{Zaključak}

\begin{itemize}
\item
\item
\item
\end{itemize}

\end{frame}

%------------------------------------------------

\section{Literatura}

\begin{frame}
\frametitle{Literatura}
\footnotesize{
\begin{thebibliography}{99}

\bibitem[]{p1} Ime Prezime pisca (godina)
\newblock \small{\textbf{Ime knjige} Mesto, tekst, godina.}

\bibitem[]{p1} Ime Prezime pisca (od-do)
\newblock \small{\textbf{Naziv dela}, on-line at: http://adresa.org/.}

\end{thebibliography}
}

\end{frame}
%------------------------------------------------

%\begin{frame}
%\Huge{\centerline{Hvala na pažnji!}}
%\end{frame}


\end{document} 
