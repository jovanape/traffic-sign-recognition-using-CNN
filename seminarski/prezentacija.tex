\documentclass{beamer}
\mode<presentation> {
\usetheme{Antibes}
\definecolor{zelena}{rgb}{.9,.74,.00}
\usecolortheme[named=zelena]{structure}
}

\usepackage[T1]{fontenc}

\usepackage{hyperref} 
\usepackage{graphicx}
\usepackage{color}
\usepackage[english,serbian]{babel}
\usepackage[utf8]{inputenc}
\usepackage{listings}


\def\d{{\fontencoding{T1}\selectfont\dj}}
\def\D{{\fontencoding{T1}\selectfont\DJ}}

\title[Prepoznavanje saobraćajnih znakova pomoću CNN]{Prepoznavanje saobraćajnih znakova pomoću CNN}

\author{Jana Jovičić, Jovana Pejkić}
\institute[Matematički fakultet]
{
\small{Prezentacija seminarskog rada \\u okviru kursa\\ Računarska inteligencija \\ Matematički fakultet\\}
\medskip
\textit{jana.jovicic755@gmail.com, jov4ana@gmail.com}
}
\date{}

\begin{document}


\begin{frame}
\titlepage
\end{frame}

%------------------------------------------------


\begin{frame}
\frametitle{Sadržaj}
\tableofcontents
\end{frame}

%------------------------------------------------



\section{Cilj rada}
\begin{frame} 
\frametitle{Cilj rada}

\begin{itemize}
\item Za bazu podataka kineskih saobracajnih znakova izvršiti što precizniju klasifikaciju
\item Implementirati CNN u programskom jeziku Python uz korišćenje Keras biblioteke
\item Isprobati nekoliko različitih arhitektura mreže
\item Uporediti dobijene rezultate i izvesti zakljlučke
\end{itemize}


%\begin{figure}
%\includegraphics[width=270pt, height=126pt]{ime_slike.jpg}
%\caption{Naslov slike}
%\end{figure}

\end{frame}

%------------------------------------------------

\section{Informacije o korišćenom skupu podataka}
\begin{frame}
\frametitle{Informacije o korišćenom skupu podataka}

\begin{itemize}
\item Baza sadrži \textbf{6164 slika} saobraćajnih znakova
\begin{itemize}
\item podeljenih u \textbf{58 klasa}
\item pri čemu \textbf{trening skup} sadrži \textbf{4170 slika}
\item a \textbf{test skup 1994 slika}
\end{itemize}
\item Zbog nejednakog broja slika (negde 5, negde 400) po klasama, korišćen je deo baze
\item Izdvojeno je \textbf{10 klasa} koje su imale priblizno jednak broj slika
\item Dobijen je \textbf{trening skup} od \textbf{1693 slika} i \textbf{test skup} od \textbf{764 slika}


% TODO TODO TODO TODO TODO
% Na slici 10 je prikazan po jedan znak iz svake klase trening skupa, zajedno sa brojem elemenata te klase.
% Podaci o test skupu, mogu se videti na slici.
% Ubaciti slike pa srediti ovo zakomentarisano!

\end{itemize}

\end{frame}

\section{Modeli}

\subsection{Model 1}
\begin{frame}
\frametitle{Model 1}

\begin{itemize}

\item Jedan od prvih modela koji je imao uspeha nad test podacima
\item Sastoji se iz:
\begin{itemize}
\item 4 konvolutivna sloja
\item 2 agregirajuca sloja
\item 2 potpuno povezana sloja
\end{itemize}

\item U svim konvolutivnim slojevima:
\begin{itemize}
\item velicina jezgra je 3x3
\item broj ltera na izlazu iz konvolucije je 32
\end{itemize}

% TODO TODO TODO TODO TODO Proveriti da li u svakom sloju?
% TODO TODO TODO TODO TODO Sta u svakom?

\item U svakom sloju se koristi \textbf{ReLU} aktivaciona funkcija
\item Agregacija se vrsi biranjem \textbf{maksimalne vrednosti} dela mape karakteristika koji je prekriven lterom

\end{itemize}


\end{frame}


\subsection{Model 1 nastavak}
\begin{frame}
\frametitle{Model 1 nastavak}

\begin{itemize}

\item Funkcijom \textbf{Dropoup()} je iskljucivan odreden broj nasumicno odabranih neurona (da bi se sprecilo preprilagodjavanje)
\item Nakon agregacija je iskljuceno 20\% neurona, pre FC sloja 50\%

\item Poslednji potpuno povezani (FC) sloj
\begin{itemize}
\item ima onoliko neurona koliko ima klasa
\item koristi softmax aktivacionu funkciju
\end{itemize}

\item Ucenje modela je sprovedeno u 30 epoha
\item Batch size je postavljen na 32
\begin{itemize}
\item sto znaci da u svakoj iteraciji uzima 32 primerka iz trening skupa koja ce biti propagirana kroz mrezu
\end{itemize}

\item Optimizacija modela je izvrsena pomocu \textbf{gradijentnog spusta}

\end{itemize}

\end{frame}

%------------------------------------------------

\subsection{Model 2}
\begin{frame}
\frametitle{Model 2}


\end{frame}

%------------------------------------------------


\subsection{Model 2 nastavak}
\begin{frame}
\frametitle{Model 2 nastavak}

\begin{itemize}
\item
\item
\item
\end{itemize}

\end{frame}

%------------------------------------------------

\subsection{Model 3}
\begin{frame}
\frametitle{Model 3}

\end{frame}

%------------------------------------------------


\section{Uporedjivanje modela}
\begin{frame}
\frametitle{Uporedjivanje modela}


\begin{block}{Primer koda}

\begin{semiverbatim}
-- neki kod
-- ...
-- ...
-- ...
\end{semiverbatim}
\end{block}

\end{frame}

%------------------------------------------------

\subsection{Statistike}
\begin{frame}
\frametitle{Statistike}


\end{frame}

%------------------------------------------------

\subsection{Grafik}
\begin{frame}
\frametitle{Grafik}

\end{frame}

%--------------------------------------------------

\section{Zaključak}

\begin{frame}
\frametitle{Zaključak}

\begin{itemize}
\item
\item
\item
\end{itemize}

\end{frame}

%------------------------------------------------

\section{Literatura}

\begin{frame}
\frametitle{Literatura}
\footnotesize{
\begin{thebibliography}{99}

\bibitem[]{p1} Ime Prezime pisca (godina)
\newblock \small{\textbf{Ime knjige} Mesto, tekst, godina.}

\bibitem[]{p1} Ime Prezime pisca (od-do)
\newblock \small{\textbf{Naziv dela}, on-line at: http://adresa.org/.}

\end{thebibliography}
}

\end{frame}
%------------------------------------------------

\begin{frame}
\Huge{\centerline{Hvala na pažnji!}}
\end{frame}


\end{document} 
